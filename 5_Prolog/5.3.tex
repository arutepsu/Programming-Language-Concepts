\section*{Aufgabe 5c -- Bahnverbindungen (Rekursion mit Umsteigen)}

Gegeben sind Zugverbindungen in Form von Fakten \texttt{zug/4}:
\texttt{zug(From, Dep, To, Arr)}. Dabei sind die Zeiten im Format \texttt{HH.MM}
notiert (z.\,B. \texttt{08.39} f\"ur 08:39 Uhr). Ziel ist ein Pr\"adikat
\texttt{verbindung/4}, das pr\"uft, ob zwischen zwei St\"adten nach einer gegebenen
Startzeit eine Verbindung existiert (inklusive Umsteigen). Beim Umsteigen muss
gelten: \texttt{Abfahrtszeit > Ankunftszeit}.

\subsection*{Faktenbasis}

\begin{verbatim}
zug(konstanz, 08.39, offenburg, 10.59).
zug(konstanz, 08.39, karlsruhe, 11.49).
zug(konstanz, 09.06, singen, 09.31).
zug(singen,   09.36, stuttgart, 11.32).
zug(offenburg,11.28, mannheim, 12.24).
zug(karlsruhe,12.06, mainz,    13.47).
zug(stuttgart,11.51, mannheim, 12.28).
zug(mannheim, 12.39, mainz,    13.18).
\end{verbatim}

\subsection*{Implementierung von \texttt{verbindung/4}}

Zur Zeitverarbeitung wird \texttt{HH.MM} in Minuten umgerechnet, indem die
Nachkommastellen als Minuten interpretiert werden (z.\,B. \texttt{08.39} $\to$
$8*60+39$).

\begin{verbatim}
% --- Zeitumrechnung: HH.MM -> Minuten seit 00:00 ---
time_minutes(T, Minutes) :-
    H is floor(T),
    M is round((T - H) * 100),     % .39 -> 39 Minuten
    Minutes is H * 60 + M.

after_or_equal(T1, T2) :-
    time_minutes(T1, M1),
    time_minutes(T2, M2),
    M1 >= M2.

after_strict(T1, T2) :-
    time_minutes(T1, M1),
    time_minutes(T2, M2),
    M1 > M2.

% --- Verbindung: Reiseplan ist Liste von zug/4-Strukturen ---
verbindung(From, StartAfter, To, Plan) :-
    verbindung_(From, StartAfter, To, [From], Plan).

% Direktverbindung: ein Zug reicht
verbindung_(From, StartAfter, To, Visited, [zug(From,Dep,To,Arr)]) :-
    zug(From, Dep, To, Arr),
    after_or_equal(Dep, StartAfter),
    \+ member(To, Visited).

% Umsteigen: erster Zug zu einer Zwischenstation, danach weiter
verbindung_(From, StartAfter, To, Visited, [zug(From,Dep,Mid,Arr)|Rest]) :-
    zug(From, Dep, Mid, Arr),
    after_or_equal(Dep, StartAfter),
    \+ member(Mid, Visited),
    verbindung_(Mid, Arr, To, [Mid|Visited], Rest),
    % Umsteigenbedingung: naechste Abfahrt > vorherige Ankunft
    Rest = [zug(Mid,Dep2,_,_)|_],
    after_strict(Dep2, Arr).
\end{verbatim}

\subsection*{Beispielabfrage}

\begin{verbatim}
?- verbindung(konstanz, 8.00, mainz, Reiseplan).
\end{verbatim}

Die Abfrage liefert nacheinander alle m\"oglichen Reisepl\"ane nach 8 Uhr als
Liste von Teilstrecken \texttt{zug(...)}. Die Reihenfolge der ausgegebenen
L\"osungen ergibt sich aus der Tiefensuche von Prolog (links-nach-rechts) und
Backtracking \"uber die \texttt{zug/4}-Fakten.
