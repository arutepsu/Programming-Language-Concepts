\section*{Aufgabe 5b -- Berechnung der Summe einer Liste in Prolog}

In dieser Aufgabe wird ein Prolog-Pr\"adikat \texttt{sum/2} implementiert, das die
Summe einer Liste von Zahlen berechnet. Die L\"osung verwendet Rekursion, wie in
der Aufgabenstellung gefordert.

\subsection*{Definition des Pr\"adikats}

Das Pr\"adikat besteht aus einem Basisfall und einem Rekursionsfall.

\bigskip

\begin{verbatim}
sum([], 0).
sum([H|T], S) :-
    sum(T, S1),
    S is H + S1.
\end{verbatim}

\bigskip

\subsection*{Erkl\"arung}

\begin{itemize}
  \item Der Basisfall legt fest, dass die Summe der leeren Liste \texttt{0} ist.
  \item Im Rekursionsfall wird die Summe berechnet, indem das erste Element der
        Liste zur Summe des restlichen Listenteils addiert wird.
  \item Der Operator \texttt{is} dient zur Auswertung arithmetischer Ausdr\"ucke.
\end{itemize}

\subsection*{Beispielanfragen}

\begin{verbatim}
?- sum([1,2,3,4], S).
S = 10.
\end{verbatim}

\begin{verbatim}
?- sum([], S).
S = 0.
\end{verbatim}

\begin{verbatim}
?- sum([5], S).
S = 5.
\end{verbatim}

\subsection*{Hinweise}

Das Pr\"adikat \texttt{sum/2} ist nicht bidirektional, da arithmetische Berechnungen
mit \texttt{is/2} nur mit instanziierten Operanden m\"oglich sind. Die Rekursion
verarbeitet die Liste von links nach rechts und endet beim Erreichen der leeren
Liste.
